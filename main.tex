 \input{article}
 
\usepackage{caption}


\begin{document}
\begin{center}
    
    \normalsize{Федеральное государственное автономное образовательное учреждение высшего образования}
    
    \textbf{НАЦИОНАЛЬНЫЙ ИССЛЕДОВАТЕЛЬСКИЙ УНИВЕРСИТЕТ \\ <<МОСКОВСКИЙ ФИЗИКО-ТЕХНИЧЕСКИЙ ИНСТИТУТ>>}
    \vspace{13ex}
    
    \textbf{Вопрос по выбору по теме << Туннельный эффект и поляризация света при полном внутреннем отражении света>>}
    \vspace{40ex}
\end{center}
\begin{flushright}
    \normalsize{Выполнил: Сидельников Станислав Игоревич \\ студент Б01-902\\}
\end{flushright}
    
\vfill
    
\begin{center}
Долгопрудный, 2020
\end{center}

\thispagestyle{empty} % выключаем отображение номера для этой страницы

\newpage

	\section{Аннотация}
	
	В данном вопросе рассматривается поведение электромагнитной волны (свет) при явлении полного внутреннего отражения на границе двух сред. 
	
	Известно, что при явлении полного внутреннего отражения на границе двух сред свет не проникает во вторую среду. На самом же деле происходит аналог туннельного эффекта в квантовой механике. Свет проникает во вторую среду в виде быстро затухающей волны, как мы увидим, глубина проникновения представлет собой величину порядка длины световой волны.
	
	Если же на расстоянии глубины проникновения расположить границу \textit{третьей} более плотной среды, то проникнувшая волна преломится на ее границе и проникнет в третью среду, продолжая распространяться в ней.
	
	\section{Теоретические сведения}
	
	Рассмотрим явления, происходящие  при падении волны на границу двух сред. 
	
	\begin{figure}[h]
		%%% [width = \linewidth] - specify line width for image
		\includegraphics[width=\linewidth]{frenel.jpg}
	\end{figure}

	Из граничных условий для магнитного поля H и электрического поля E получаем:
	
	\[ \varphi = \varphi_{1}, \ \frac{\sin{ \varphi}} { \sin{ \varphi_{2}} } = \frac{n_{2}}{n_{1}} = n \]
	
	Также из формул Френеля получаем коэффициенты отражения и преломления для случая s- и p- поляризации (r - отраженная, d - преломленная волна):
	
	
	\[ R_{\perp} = - \frac{ \sin{(\varphi - \psi)}}{\sin{ (\varphi + \psi)}} \]
	
	\[ R_{\parallel} = \substack{ - \\ (+)} \frac{ \tg{(\varphi - \psi)}}{\tg{ (\varphi + \psi)}}  \]
	
	\[ D_{\perp} = - \frac{2  \sin{ \psi} \cos{\psi}} {\sin{ (\varphi + \psi)}} \]
	
		\[ D_{\parallel} = - \frac{2  \sin{ \psi} \cos{\psi}} {\sin{ (\varphi + \psi)} \cos(\varphi - \psi) } \]
		
	Данные формулы описывают связь амплитуд падающей, отраженной и преломленной волны на границе двух сред. Знак + для $ R_{\parallel} $ означает случай $ \varphi + \psi  > \pi / 2 $

	\section{Явление полного внутреннего отражения, затухающая волна}
	
	\begin{figure}[h]
		%%% [width = \linewidth] - specify line width for image
		\includegraphics[width=\linewidth]{reflection.jpg}
	\end{figure}

	При условии $ \sin{\psi} =  \frac{\sin{\varphi}}{n} > 1$ косинус угла
	$ \psi $ получается чисто мнимым числом. Кажется, что в таком случае не происходить преломления волны, вся волна отражается в изначальную среду.
	
	На самом же деле волна проникает во вторую среду, но при этом она является неоднородной и быстро затухает. Для этого рассмотрим выражение для волны во второй среде.
	
	Уравнение плоской волны во второй среде в общем случае:
	
	\[ E_{2} = E_{0} \ exp(i (\omega t - k_{2} x \sin{\psi} - k_{2} z \cos{\psi} )) \]
	
	Используем здесь выражение для косинуса $\psi$:
	$\cos{ \psi } = - \frac{i}{n} \sqrt{ \sin^{2}{\varphi}- n^{2} } $
	
	Принципиально важно здесь взять косинус со знаком минус, о чем упомянем позже.
	
	Тогда наше уравнение сведется к следующему: \[ E_{2} = E_{0} \ exp (  i (\omega t - k_{2} x \frac{ \sin{\varphi}}{n} )) \ exp ( - z  \frac{ k_{2} } {n} \sqrt{ \sin^{2}{\varphi}- n^{2} } ) \]
	
	Откуда мы видим, что уравнение представляет собой неоднородную волну с зависящей от времени амплитудой, а также затухающий множитель (здесь z - направление распространение волны во вторую среду). Теперь видно, что если взять косинус из под корня со знаком плюс, то мы получим бесконечно растущее решение для проникающей волны, что не соответствует физическим представлениям. Оценим, на какую глубину данная волна проникает во вторую среду.

	
	Амплитуда уменьшается в e раз на глубине	
	\[h = \frac{\lambda}{2 \pi \sqrt{\sin^{2}{\varphi}  - n^2} } \simeq 0.4 \lambda \ \text{ (Взят $\varphi$ = $\pi / 4 \ > \  \varphi_{\text{крит}} )$} \]
	
	Таким образом, глубина проникновения сопоставима с длиной волны. Для света - порядка сотни нанометров, то есть проникающая во вторую среду световая волна действительно очень быстро затухает.
	
	
	
	Образование затуающей волны можно обосновать и с помощью принципа Гюйгенса, графически. При предельном угле падения во второй среде фронт волны распространяется паралеллельно границе падения, огибающий фронт волны перпендикулярен границе падения. В случае превышении угла критического значения, огибающего фронта не будет, и во второй среде полявляется неоднородная волна.
	
	\begin{figure}[h]
		\centering
		%%% [width = \linewidth] - specify line width for image
		\includegraphics[width=0.8\linewidth]{wave_fronts.png}
	\end{figure}
	
	\section{Энергия затухающей волны}
	
	Подставим мнимый косинус в формулы Френеля для случая p - поляризации и преобразуем их следующим образом (здесь E - падающая волна, R - отраженая, D - преломленная):
	
	\[  D_{\parallel} = E_{\parallel} \cdot \frac {2 n \cos{\varphi}} { \cos{ \varphi  - \text{i} \sqrt{\sin^{2}{\varphi}  - n^2}  }} \]
	
	\[  R_{\parallel} = E_{\parallel} \cdot \frac { n^{2} \cos{ \varphi  + \text{i} \sqrt{\sin^{2}{\varphi}  - n^2}  } } { n^{2} \cos{ \varphi  - \text{i} \sqrt{\sin^{2}{\varphi}  - n^2}  } } \]
	
	Видно, что $R_{\parallel} = E_{\parallel} $
	
	В то же время, например, при полном внутреннем отражении $ D_{\parallel}  = 2 E_{\parallel} $, соответственно. Нарушения закона сохранения энергии нет, дело в том, что формулы Френеля верны для случая \textit{установившегося процесса}, то есть для \textit{монохроматического поля}. В таком случае среднее изменение энергии волны во второй среде за период должно быть нулевым.
	
	Рассмотрим среднее значение вектора Пойнтинга во второй среде, а точнее его проекции на ось z  и ось x для случая p - поляризации.
	
	В этом случае для прошедшей волны,где $ D $ - амплитуда волны:
	
	\[E_{x}^{(d)} = E_{2} \cos{ \psi}, \ B_{y}^{(d)} = n E_{2}, \ E_{z}^{(d)} = E_{2} \sin{\psi} \]
	
	Для магнитного поля: 
	
	\[ B^{(d)} = B_{y}^{(d)}  = n E_{2} \]
	
	Тогда преобразуя уравнения получаем:
	
	\[E_{x}^{(d)} = p E_{0}^{2} \ exp \ [i (\omega t - k_{2x} \sin{\varphi} / n - \pi / 2)  ]  exp[- \frac{z}{h} ] \]
	
		\[E_{z}^{(d)} = \sin{\psi} E_{0}^{2} \ exp \ [i (\omega t - k_{2x} \sin{\varphi} / n)  ]  exp[- \frac{z}{h} ] \]
		
			\[B_{y}^{(d)} = n E_{0}^{2} \ exp \ [i (\omega t - k_{2x} \sin{\varphi} / n)  ]  exp[- \frac{z}{h} ] \]
			
	Где p = $ \sqrt{\sin^{2}{\varphi}  - n^2} $, а h - характерная глубина проникновения поля.
			
	Посчитаем теперь среднее значение вектора Пойнтинга в проекции на z и x:
	
	\[\bar{S_{z}} = 0 \]
	
		\[S_{x} = \frac{c}{4 \pi } \cdot E_{0}^{2} \ exp[2 i (\omega t - k_{2x} \sin{\varphi} / n) ]  \ exp[- 2 \frac{z}{h} \sin{\varphi}]\]
		
		\[  \bar{S_{x}} =  \frac{c}{4 \pi } \cdot \frac{E_{0}^{2}}{2}   \ exp[- 2 \frac{z}{h}] \sin{\varphi}  \]
		
		Усреднение по времени проекции вдоль Z дает ноль, так как среднее значение косинуса и синуса за период ноль, в случае с проекцией вдоль x, получаем в действительное части вектора Пойнтинга косинус двойного угла, который при усреднении дает ненулевое значение. Таким образом, во второй среде действительно имеется энергия, средний запас которой со временем не меняется, а средний поток параллелен оси x.
		
		\begin{figure}[h]
			\centering
			%%% [width = \linewidth] - specify line width for image
			\includegraphics[width=0.5\linewidth]{wave_axes.png}
		\end{figure}
	
	
	\section{Образование туннелированной волны}
	
	Поместим теперь на небольшом расстоянии от границы двух сред третью среду, у которой показатель преломления не меньше, чем у первой среды. Если волна из зазора коснется поверхности третьей среды, то, преломившись, она будет распространяться в этой среде. Это явление прохождения волны через запрещенную область называется \textit{туннельным эффектом}.
	
	\begin{figure}[h]
		\centering
		%%% [width = \linewidth] - specify line width for image
		\includegraphics[width=0.4\linewidth]{tunneling.png}
	\end{figure}
	
	При  достаточно малой толщине зазора  неоднородная волна достигает второй границы еще не очень ослабленной, вступив с воздушного зазора во вторую среду, волна снова превращается в однородную и может быть обнаружена обычными средствами. Опыт такого рода был поставлен еще Ньютоном. Опыт заключался в следующем: Ньютон прижимал к гипотенузной грани прямоугольной призмы другую призму сферической формы (см. рис). Оказалось, что свет проходит в призму не только в месте оптического контакта (при этом угол падения больше предельного), но и в небольшом кольце вокруг него -  там, где толщина воздушного промежутка сравнима с длиной волны. При наблюдении в белом свете внешний край кольца имел крановатый оттенок, что связано с большей длиной волны красных лучей.
	
	\newpage
	\begin{figure}[h]
		\centering
		%%% [width = \linewidth] - specify line width for image
		\includegraphics[width=0.4\linewidth]{newton.png}
		\caption*{Опыт Ньютона}
	\end{figure}

	\begin{figure}[h!]
		\centering
		\begin{minipage}{0.4\linewidth}
	\includegraphics[width=\linewidth]{Tunnel1.jpg}
	\caption*{Полное отражение}
		\end{minipage}
		%%% [width = \linewidth] - specify line width for image
		\begin{minipage}{0.4\linewidth}
			\includegraphics[width=\linewidth]{Tunnel2.jpg}
			\caption*{Недостаточно малый зазор}
		\end{minipage}
	
		\begin{minipage}{0.4\linewidth}
		\includegraphics[width=\linewidth]{Tunnel3.jpg}
		\caption*{Эффект туннелирования, зазор достаточно мал}
	\end{minipage}
	\end{figure}





	
	
	
%	Из этих формул видно, что $\abs{r} = \abs{E}$
	
	
	
	\section{Изменение фазы волны при полном внутреннем отражении}
	
	При полном внутреннем отражении коэффициенты $ R_{\parallel} $ и $ R_{\perp} $, вообще говоря, \textit{комплексны}. Рассмотрим, например, случай p - поляризации с коэффициентов отражения 
	\[ R_{\parallel} = E_{\parallel} \cdot \frac { n^{2} \cos{ \varphi  + \text{i} \sqrt{\sin^{2}{\varphi}  - n^2}  } } { n^{2} \cos{ \varphi  - \text{i} \sqrt{\sin^{2}{\varphi}  - n^2}  } } \]

	Теперь, когда величины мнимые, между отраженной и падающей волной есть сдвиг по фазе $ \delta_{\parallel} $.
	
	Действительно, распишем связь падающей и отраженной волны. Падающая:
	
	\[E_{1} = E_{0} \ exp[ i (\omega t - k_{x} x - k_{z} z)] \]
	
	Отраженная:
	
	\begin{multline*}
		E_{1}^{'} =  R_{\parallel} = E_{1} \cdot \frac { n^{2} \cos{ \varphi  + \text{i} \sqrt{\sin^{2}{\varphi}  - n^2}  } } { n^{2} \cos{ \varphi  - \text{i} \sqrt{\sin^{2}{\varphi}  - n^2}  } } = \\ 
		= E_{0} \ exp[ i (\omega t - k_{x} x - k_{z} z)] \cdot exp[\text{i} \arctg{\frac{2 \cos{\varphi}   \sqrt{\sin^{2}{\varphi}  - n^2}   }{\cos{\varphi^{2} - \sin{\varphi}^{2} - n^{2}}}  } ] 
		\end{multline*}
		
		\[ \tg{\frac{\delta_{\parallel}}{2}} = \frac{ \sqrt{\sin^{2}{\varphi}  - n^2}}{n^{2} \cos{\varphi}}\]
	
	Теперь отметим, что сдвиг по фазе для перпендикулярного случая s - поляризации имеет иное значение:
	
		\[ \tg{\frac{\delta_{\perp}}{2}} = \frac{ \sqrt{\sin^{2}{\varphi}  - n^2}}{ \cos{\varphi}}\]
		
		Поэтому при отражении данной волны перпендикулярные и продольные составляющие, вообще говоря, будут \textit{сдвинуты относительно друг друга по фазе.} 
		
		Рассмотрим сдвиг по фазе, проанализировав разность $ \delta_{\parallel} $ и$  \delta_{\perp} $ 
		
		\[ \tg{\frac{\delta}{2}} = \tg{\left( \frac{\delta_{\parallel} - \delta_{\perp}}{2} \right) } = \frac{\cos{\varphi}  \sqrt{\sin^{2}{\varphi}  - n^2} }{ \sin^{2}{\varphi}} \]
		
		Это выражение положительно, значит, \textbf{$E_{\parallel}$}
		опережает по фазе  \textbf{$E_{\perp}$}
		
		Исследуя данное выражение на экстремум по переменной $\varphi$, получаем, что максимальная разность фаз получается при $\cos{\varphi}  = \sqrt{ (1 - n^{2}) / (1 + n^{2})} $
	 	Для получения сдвига фаз, например, на $\pi / 2$, подставляя это значение в уравнение выше, получаем, что $ n = 0.414 $. При этом показатель преломления оптически более плотной среды относительно менее плотной будет равен $n^{'} = \frac{1}{n} = 2.41 $. Только алмаз обладает достаточным показателем преломления для получения разности фаз в 90 при однакратном отражении, однако в таком случае применяется многократные отражения. Например, так устроен ромб Френеля, использующийся для поляризации света.
	 	
	 	\newpage
	 	
	 	\begin{figure}[h]
	 		\centering
	 		%%% [width = \linewidth] - specify line width for image
	 		\includegraphics[width=0.4\linewidth]{frenel_exp.png}
	 		\caption*{Ромб Френеля}
	 	\end{figure}
 	
 	Для стекла было вычислено, что при угле падения $\varphi = 54^{\circ}37'$ разность сдвиг по фазе $\delta$ составяет $45^{\circ}$. Таким образом, при двухкратном отражении на выходе свет окажется сдвинутым по фазе на $90^{\circ}$. 
	 	
	 \section{Список литературы}
	 
	 \begin{itemize}
	 	\item Общий курс физики, Т4: Оптика, Сивухин Д.В
	 	\item Туннельный эффект при полном внутреннем отражении света, А.В. Шушарин
	 	
	 \end{itemize}

\end{document}
